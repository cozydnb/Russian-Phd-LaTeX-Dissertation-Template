%&preformat-disser
\RequirePackage[l2tabu,orthodox]{nag} % Раскомментировав, можно в логе получать рекомендации относительно правильного использования пакетов и предупреждения об устаревших и нерекомендуемых пакетах
% Формат А4, 14pt (ГОСТ Р 7.0.11-2011, 5.3.6)
\documentclass[a4paper,14pt,oneside,openany]{memoir}

%%%%%%%%%%%%%%%%%%%%%%%%%%%%%%%%%%%%%%%%%%%%%%%%%%%%%%%%%%%%%%%%%%%%%%%%%%%%%%%%
%%%% Файл упрощённых настроек шаблона, общих для диссертации и автореферата %%%%
%%%%%%%%%%%%%%%%%%%%%%%%%%%%%%%%%%%%%%%%%%%%%%%%%%%%%%%%%%%%%%%%%%%%%%%%%%%%%%%%

%%% Режим черновика %%%
\makeatletter
\@ifundefined{c@draft}{
  \newcounter{draft}
  \setcounter{draft}{0}  % 0 --- чистовик (максимальное соблюдение ГОСТ)
                         % 1 --- черновик (отклонения от ГОСТ, но быстрая
                         %       сборка итоговых PDF)
}{}
\makeatother

%%% Пометки в тексте %%%
\makeatletter
\@ifundefined{c@showmarkup}{
  \newcounter{showmarkup}
  \setcounter{showmarkup}{0}  % 0 --- скрыть пометки
                              % 1 --- показывать пометки
}{}
\makeatother

%%% Использование в pdflatex шрифтов не по-умолчанию %%%
\makeatletter
\@ifundefined{c@usealtfont}{
  \newcounter{usealtfont}
  \setcounter{usealtfont}{1}    % 0 --- шрифты на базе Computer Modern
                                % 1 --- использовать пакет pscyr, при его
                                %       наличии
                                % 2 --- использовать пакет XCharter, при наличии
                                %       подходящей версии
}{}
\makeatother

%%% Использование в xelatex и lualatex семейств шрифтов %%%
\makeatletter
\@ifundefined{c@fontfamily}{
  \newcounter{fontfamily}
  \setcounter{fontfamily}{1}  % 0 --- CMU семейство. Используется как fallback;
                              % 1 --- Шрифты от MS (Times New Roman и компания)
                              % 2 --- Семейство Liberation
}{}
\makeatother

%%% Библиография %%%
\makeatletter
\@ifundefined{c@bibliosel}{
  \newcounter{bibliosel}
  \setcounter{bibliosel}{1}   % 0 --- встроенная реализация с загрузкой файла
                              %       через движок bibtex8;
                              % 1 --- реализация пакетом biblatex через движок
                              %       biber
}{}
\makeatother

%%% Вывод типов ссылок в библиографии %%%
\makeatletter
\@ifundefined{c@mediadisplay}{
  \newcounter{mediadisplay}
  \setcounter{mediadisplay}{2}   % 0 --- не делать ничего; надписи [Текст] и
                                 %       [Эл. ресурс] будут выводиться только в ссылках с
                                 %       заполненным полем `media`;
                                 % 1 --- автоматически добавлять надпись [Текст] к ссылкам с
                                 %       незаполненным полем `media`; таким образом, у всех
                                 %       источников будет указан тип, что соответствует
                                 %       требованиям ГОСТ
                                 % 2 --- автоматически удалять надписи [Текст], [Эл. Ресурс] и др.;
                                 %       не соответствует ГОСТ
                                 % 3 --- автоматически удалять надпись [Текст];
                                 %       не соответствует ГОСТ
                                 % 4 --- автоматически удалять надпись [Эл. Ресурс];
                                 %       не соответствует ГОСТ
}{}
\makeatother

%%% Предкомпиляция tikz рисунков для ускорения работы %%%
\makeatletter
\@ifundefined{c@imgprecompile}{
  \newcounter{imgprecompile}
  \setcounter{imgprecompile}{0}   % 0 --- без предкомпиляции;
                                  % 1 --- пользоваться предварительно
                                  %       скомпилированными pdf вместо генерации
                                  %       заново из tikz
}{}
\makeatother
            % общие настройки шаблона
\input{common/packages}         % Пакеты общие для диссертации и автореферата
\synopsisfalse                      % Этот документ --- не автореферат
\input{Dissertation/dispackages}    % Пакеты для диссертации
\input{Dissertation/userpackages}   % Пакеты для специфических пользовательских задач

%%%%%%%%%%%%%%%%%%%%%%%%%%%%%%%%%%%%%%%%%%%%%%%%%%%%%%%%%%%%%%%%%%%%%%%%%%%%%%%%
%%%% Файл упрощённых настроек шаблона, общих для диссертации и автореферата %%%%
%%%%%%%%%%%%%%%%%%%%%%%%%%%%%%%%%%%%%%%%%%%%%%%%%%%%%%%%%%%%%%%%%%%%%%%%%%%%%%%%

%%% Режим черновика %%%
\makeatletter
\@ifundefined{c@draft}{
  \newcounter{draft}
  \setcounter{draft}{0}  % 0 --- чистовик (максимальное соблюдение ГОСТ)
                         % 1 --- черновик (отклонения от ГОСТ, но быстрая
                         %       сборка итоговых PDF)
}{}
\makeatother

%%% Пометки в тексте %%%
\makeatletter
\@ifundefined{c@showmarkup}{
  \newcounter{showmarkup}
  \setcounter{showmarkup}{0}  % 0 --- скрыть пометки
                              % 1 --- показывать пометки
}{}
\makeatother

%%% Использование в pdflatex шрифтов не по-умолчанию %%%
\makeatletter
\@ifundefined{c@usealtfont}{
  \newcounter{usealtfont}
  \setcounter{usealtfont}{1}    % 0 --- шрифты на базе Computer Modern
                                % 1 --- использовать пакет pscyr, при его
                                %       наличии
                                % 2 --- использовать пакет XCharter, при наличии
                                %       подходящей версии
}{}
\makeatother

%%% Использование в xelatex и lualatex семейств шрифтов %%%
\makeatletter
\@ifundefined{c@fontfamily}{
  \newcounter{fontfamily}
  \setcounter{fontfamily}{1}  % 0 --- CMU семейство. Используется как fallback;
                              % 1 --- Шрифты от MS (Times New Roman и компания)
                              % 2 --- Семейство Liberation
}{}
\makeatother

%%% Библиография %%%
\makeatletter
\@ifundefined{c@bibliosel}{
  \newcounter{bibliosel}
  \setcounter{bibliosel}{1}   % 0 --- встроенная реализация с загрузкой файла
                              %       через движок bibtex8;
                              % 1 --- реализация пакетом biblatex через движок
                              %       biber
}{}
\makeatother

%%% Вывод типов ссылок в библиографии %%%
\makeatletter
\@ifundefined{c@mediadisplay}{
  \newcounter{mediadisplay}
  \setcounter{mediadisplay}{2}   % 0 --- не делать ничего; надписи [Текст] и
                                 %       [Эл. ресурс] будут выводиться только в ссылках с
                                 %       заполненным полем `media`;
                                 % 1 --- автоматически добавлять надпись [Текст] к ссылкам с
                                 %       незаполненным полем `media`; таким образом, у всех
                                 %       источников будет указан тип, что соответствует
                                 %       требованиям ГОСТ
                                 % 2 --- автоматически удалять надписи [Текст], [Эл. Ресурс] и др.;
                                 %       не соответствует ГОСТ
                                 % 3 --- автоматически удалять надпись [Текст];
                                 %       не соответствует ГОСТ
                                 % 4 --- автоматически удалять надпись [Эл. Ресурс];
                                 %       не соответствует ГОСТ
}{}
\makeatother

%%% Предкомпиляция tikz рисунков для ускорения работы %%%
\makeatletter
\@ifundefined{c@imgprecompile}{
  \newcounter{imgprecompile}
  \setcounter{imgprecompile}{0}   % 0 --- без предкомпиляции;
                                  % 1 --- пользоваться предварительно
                                  %       скомпилированными pdf вместо генерации
                                  %       заново из tikz
}{}
\makeatother
      % Упрощённые настройки шаблона

\input{common/newnames}         % Новые переменные, для всего проекта

\input{common/data}             % Основные сведения
\input{common/fonts}            % Определение шрифтов (частичное)
\input{common/styles}           % Стили общие для диссертации и автореферата
\input{Dissertation/disstyles}  % Стили для диссертации
\input{Dissertation/userstyles} % Стили для специфических пользовательских задач

%%% Библиография. Выбор движка для реализации %%%
% Здесь только проверка установленного ключа. Сама настройка выбора движка
% размещена в common/setup.tex
\ifnumequal{\value{bibliosel}}{0}{%
    \input{biblio/predefined}   % Встроенная реализация с загрузкой файла через движок bibtex8
}{
    %%% Реализация библиографии пакетами biblatex и biblatex-gost с использованием движка biber %%%

\usepackage{csquotes} % biblatex рекомендует его подключать. Пакет для оформления сложных блоков цитирования.
%%% Загрузка пакета с основными настройками %%%
\makeatletter
\ifnumequal{\value{draft}}{0}{% Чистовик
\usepackage[%
backend=biber,% движок
bibencoding=utf8,% кодировка bib файла
sorting=none,% настройка сортировки списка литературы
style=gost-numeric,% стиль цитирования и библиографии (по ГОСТ)
language=autobib,% получение языка из babel/polyglossia, default: autobib % если ставить autocite или auto, то цитаты в тексте с указанием страницы, получат указание страницы на языке оригинала
autolang=other,% многоязычная библиография
clearlang=true,% внутренний сброс поля language, если он совпадает с языком из babel/polyglossia
defernumbers=true,% нумерация проставляется после двух компиляций, зато позволяет выцеплять библиографию по ключевым словам и нумеровать не из большего списка
sortcites=true,% сортировать номера затекстовых ссылок при цитировании (если в квадратных скобках несколько ссылок, то отображаться будут отсортированно, а не абы как)
doi=false,% Показывать или нет ссылки на DOI
isbn=false,% Показывать или нет ISBN, ISSN, ISRN
]{biblatex}[2016/09/17]
\ltx@iffilelater{biblatex-gost.def}{2017/05/03}%
{\toggletrue{bbx:gostbibliography}%
\renewcommand*{\revsdnamepunct}{\addcomma}}{}
}{%Черновик
\usepackage[%
backend=biber,% движок
bibencoding=utf8,% кодировка bib файла
sorting=none,% настройка сортировки списка литературы
% defernumbers=true, % откомментируйте, если требуется правильная нумерация ссылок на литературу в режиме черновика. Замедляет сборку
]{biblatex}[2016/09/17]%
}

\makeatother
\AtEveryBibitem{%
  \clearfield{note}%
}
\providebool{blxmc} % biblatex version needs and has MakeCapital workaround
\boolfalse{blxmc} % setting our new boolean flag to default false
\ifxetexorluatex
\else
% Исправление случая неподдержки знака номера в pdflatex
    \DefineBibliographyStrings{russian}{number={\textnumero}}

% Исправление случая отсутствия прописных букв в некоторых случаях
% https://github.com/plk/biblatex/issues/960#issuecomment-596658282
    \ifdefmacro{\ExplSyntaxOn}{}{\usepackage{expl3}}
    \makeatletter
    \ltx@ifpackagelater{biblatex}{2020/02/23}{
    % Assuming this version of biblatex defines MakeCapital correctly
    }{
        \ltx@ifpackagelater{biblatex}{2019/12/01}{
            % Assuming this version of biblatex defines MakeCapital incorrectly
            \usepackage{expl3}[2020/02/25]
            \@ifpackagelater{expl3}{2020/02/25}{
                \booltrue{blxmc} % setting our new boolean flag to true
            }{}
        }{}
    }
    \makeatother
    \ifblxmc
        \typeout{Assuming this version of biblatex defines MakeCapital
        incorrectly}
        \usepackage{xparse}
        \makeatletter
        \ExplSyntaxOn
        \NewDocumentCommand \blx@maketext@lowercase {m}
          {
            \text_lowercase:n {#1}
          }

        \NewDocumentCommand \blx@maketext@uppercase {m}
          {
            \text_uppercase:n {#1}
          }

        \RenewDocumentCommand \MakeCapital {m}
          {
            \text_titlecase_first:n {#1}
          }
        \ExplSyntaxOff

        \protected\def\blx@biblcstring#1#2#3{%
          \blx@begunit
          \blx@hyphenreset
          \blx@bibstringsimple
          \lowercase{\edef\blx@tempa{#3}}%
          \ifcsundef{#2@\blx@tempa}
            {\blx@warn@nostring\blx@tempa
             \blx@endnounit}
            {#1{\blx@maketext@lowercase{\csuse{#2@\blx@tempa}}}%
             \blx@endunit}}

        \protected\def\blx@bibucstring#1#2#3{%
          \blx@begunit
          \blx@hyphenreset
          \blx@bibstringsimple
          \lowercase{\edef\blx@tempa{#3}}%
          \ifcsundef{#2@\blx@tempa}
            {\blx@warn@nostring\blx@tempa
             \blx@endnounit}
            {#1{\blx@maketext@uppercase{\csuse{#2@\blx@tempa}}}%
             \blx@endunit}}
        \makeatother
    \fi
\fi

\ifsynopsis
\ifnumgreater{\value{usefootcite}}{0}{
    \ExecuteBibliographyOptions{autocite=footnote}
    \newbibmacro*{cite:full}{%
        \printtext[bibhypertarget]{%
            \usedriver{%
                \DeclareNameAlias{sortname}{default}%
            }{%
                \thefield{entrytype}%
            }%
        }%
        \usebibmacro{shorthandintro}%
    }
    \DeclareCiteCommand{\smartcite}[\mkbibfootnote]{%
        \usebibmacro{prenote}%
    }{%
        \usebibmacro{citeindex}%
        \usebibmacro{cite:full}%
    }{%
        \multicitedelim%
    }{%
        \usebibmacro{postnote}%
    }
}{}
\fi

%%% Подключение файлов bib %%%
\addbibresource[label=bl-external]{biblio/external.bib}
\addbibresource[label=bl-author]{biblio/author.bib}
\addbibresource[label=bl-registered]{biblio/registered.bib}

%http://tex.stackexchange.com/a/141831/79756
%There is a way to automatically map the language field to the langid field. The following lines in the preamble should be enough to do that.
%This command will copy the language field into the langid field and will then delete the contents of the language field. The language field will only be deleted if it was successfully copied into the langid field.
\DeclareSourcemap{ %модификация bib файла перед тем, как им займётся biblatex
    \maps{
        \map{% перекидываем значения полей language в поля langid, которыми пользуется biblatex
            \step[fieldsource=language, fieldset=langid, origfieldval, final]
            \step[fieldset=language, null]
        }
        \map{% перекидываем значения полей numpages в поля pagetotal, которыми пользуется biblatex
            \step[fieldsource=numpages, fieldset=pagetotal, origfieldval, final]
            \step[fieldset=numpages, null]
        }
        \map{% перекидываем значения полей pagestotal в поля pagetotal, которыми пользуется biblatex
            \step[fieldsource=pagestotal, fieldset=pagetotal, origfieldval, final]
            \step[fieldset=pagestotal, null]
        }
        \map[overwrite]{% перекидываем значения полей shortjournal, если они есть, в поля journal, которыми пользуется biblatex
            \step[fieldsource=shortjournal, final]
            \step[fieldset=journal, origfieldval]
            \step[fieldset=shortjournal, null]
        }
        \map[overwrite]{% перекидываем значения полей shortbooktitle, если они есть, в поля booktitle, которыми пользуется biblatex
            \step[fieldsource=shortbooktitle, final]
            \step[fieldset=booktitle, origfieldval]
            \step[fieldset=shortbooktitle, null]
        }
        \map{% если в поле medium написано "Электронный ресурс", то устанавливаем поле media, которым пользуется biblatex, в значение eresource.
            \step[fieldsource=medium,
            match=\regexp{Электронный\s+ресурс},
            final]
            \step[fieldset=media, fieldvalue=eresource]
            \step[fieldset=medium, null]
        }
        \map[overwrite]{% стираем значения всех полей issn
            \step[fieldset=issn, null]
        }
        \map[overwrite]{% стираем значения всех полей abstract, поскольку ими не пользуемся, а там бывают "неприятные" латеху символы
            \step[fieldsource=abstract]
            \step[fieldset=abstract,null]
        }
        \map[overwrite]{ % переделка формата записи даты
            \step[fieldsource=urldate,
            match=\regexp{([0-9]{2})\.([0-9]{2})\.([0-9]{4})},
            replace={$3-$2-$1$4}, % $4 вставлен исключительно ради нормальной работы программ подсветки синтаксиса, которые некорректно обрабатывают $ в таких конструкциях
            final]
        }
        \map[overwrite]{ % стираем ключевые слова
            \step[fieldsource=keywords]
            \step[fieldset=keywords,null]
        }
        % реализация foreach различается для biblatex v3.12 и v3.13.
        % Для версии v3.13 эта конструкция заменяет последующие 7 структур map
        % \map[overwrite,foreach={authorvak,authorscopus,authorwos,authorconf,authorother,authorparent,authorprogram}]{ % записываем информацию о типе публикации в ключевые слова
        %     \step[fieldsource=$MAPLOOP,final=true]
        %     \step[fieldset=keywords,fieldvalue={,biblio$MAPLOOP},append=true]
        % }
        \map[overwrite]{ % записываем информацию о типе публикации в ключевые слова
            \step[fieldsource=authorvak,final=true]
            \step[fieldset=keywords,fieldvalue={,biblioauthorvak},append=true]
        }
        \map[overwrite]{ % записываем информацию о типе публикации в ключевые слова
            \step[fieldsource=authorscopus,final=true]
            \step[fieldset=keywords,fieldvalue={,biblioauthorscopus},append=true]
        }
        \map[overwrite]{ % записываем информацию о типе публикации в ключевые слова
            \step[fieldsource=authorwos,final=true]
            \step[fieldset=keywords,fieldvalue={,biblioauthorwos},append=true]
        }
        \map[overwrite]{ % записываем информацию о типе публикации в ключевые слова
            \step[fieldsource=authorconf,final=true]
            \step[fieldset=keywords,fieldvalue={,biblioauthorconf},append=true]
        }
        \map[overwrite]{ % записываем информацию о типе публикации в ключевые слова
            \step[fieldsource=authorother,final=true]
            \step[fieldset=keywords,fieldvalue={,biblioauthorother},append=true]
        }
        \map[overwrite]{ % записываем информацию о типе публикации в ключевые слова
            \step[fieldsource=authorpatent,final=true]
            \step[fieldset=keywords,fieldvalue={,biblioauthorpatent},append=true]
        }
        \map[overwrite]{ % записываем информацию о типе публикации в ключевые слова
            \step[fieldsource=authorprogram,final=true]
            \step[fieldset=keywords,fieldvalue={,biblioauthorprogram},append=true]
        }
        \map[overwrite]{ % добавляем ключевые слова, чтобы различать источники
            \perdatasource{biblio/external.bib}
            \step[fieldset=keywords, fieldvalue={,biblioexternal},append=true]
        }
        \map[overwrite]{ % добавляем ключевые слова, чтобы различать источники
            \perdatasource{biblio/author.bib}
            \step[fieldset=keywords, fieldvalue={,biblioauthor},append=true]
        }
        \map[overwrite]{ % добавляем ключевые слова, чтобы различать источники
            \perdatasource{biblio/registered.bib}
            \step[fieldset=keywords, fieldvalue={,biblioregistered},append=true]
        }
        \map[overwrite]{ % добавляем ключевые слова, чтобы различать источники
            \step[fieldset=keywords, fieldvalue={,bibliofull},append=true]
        }
%        \map[overwrite]{% стираем значения всех полей series
%            \step[fieldset=series, null]
%        }
        \map[overwrite]{% перекидываем значения полей howpublished в поля organization для типа online
            \step[typesource=online, typetarget=online, final]
            \step[fieldsource=howpublished, fieldset=organization, origfieldval]
            \step[fieldset=howpublished, null]
        }
    }
}

\ifnumequal{\value{mediadisplay}}{1}{
    \DeclareSourcemap{
        \maps{%
            \map{% использование media=text по умолчанию
                \step[fieldset=media, fieldvalue=text]
            }
        }
    }
}{}
\ifnumequal{\value{mediadisplay}}{2}{
    \DeclareSourcemap{
        \maps{%
            \map[overwrite]{% удаление всех записей media
                \step[fieldset=media, null]
            }
        }
    }
}{}
\ifnumequal{\value{mediadisplay}}{3}{
    \DeclareSourcemap{
        \maps{
            \map[overwrite]{% стираем значения всех полей media=text
                \step[fieldsource=media,match={text},final]
                \step[fieldset=media, null]
            }
        }
    }
}{}
\ifnumequal{\value{mediadisplay}}{4}{
    \DeclareSourcemap{
        \maps{
            \map[overwrite]{% стираем значения всех полей media=eresource
                \step[fieldsource=media,match={eresource},final]
                \step[fieldset=media, null]
            }
        }
    }
}{}

\ifsynopsis
\else
\DeclareSourcemap{ %модификация bib файла перед тем, как им займётся biblatex
    \maps{
        \map[overwrite]{% стираем значения всех полей addendum
            \perdatasource{biblio/author.bib}
            \step[fieldset=addendum, null] %чтобы избавиться от информации об объёме авторских статей, в отличие от автореферата
        }
    }
}
\fi

\ifpresentation
% удаляем лишние поля в списке литературы презентации
% их названия можно узнать в файле presentation.bbl
\DeclareSourcemap{
    \maps{
    \map[overwrite,foreach={%
        % {{{ Список лишних полей в презентации
        address,%
        chapter,%
        edition,%
        editor,%
        eid,%
        howpublished,%
        institution,%
        key,%
        month,%
        note,%
        number,%
        organization,%
        pages,%
        publisher,%
        school,%
        series,%
        type,%
        media,%
        url,%
        doi,%
        location,%
        volume,%
        % Список лишних полей в презентации }}}
    }]{
        \perdatasource{biblio/author.bib}
        \step[fieldset=$MAPLOOP,null]
    }
    }
}
\fi

\defbibfilter{vakscopuswos}{%
    keyword=biblioauthorvak or keyword=biblioauthorscopus or keyword=biblioauthorwos
}

\defbibfilter{scopuswos}{%
    keyword=biblioauthorscopus or keyword=biblioauthorwos
}

\defbibfilter{papersregistered}{%
    keyword=biblioauthor or keyword=biblioregistered
}

%%% Убираем неразрывные пробелы перед двоеточием и точкой с запятой %%%
%\makeatletter
%\ifnumequal{\value{draft}}{0}{% Чистовик
%    \renewcommand*{\addcolondelim}{%
%      \begingroup%
%      \def\abx@colon{%
%        \ifdim\lastkern>\z@\unkern\fi%
%        \abx@puncthook{:}\space}%
%      \addcolon%
%      \endgroup}
%
%    \renewcommand*{\addsemicolondelim}{%
%      \begingroup%
%      \def\abx@semicolon{%
%        \ifdim\lastkern>\z@\unkern\fi%
%        \abx@puncthook{;}\space}%
%      \addsemicolon%
%      \endgroup}
%}{}
%\makeatother

%%% Правка записей типа thesis, чтобы дважды не писался автор
%\ifnumequal{\value{draft}}{0}{% Чистовик
%\DeclareBibliographyDriver{thesis}{%
%  \usebibmacro{bibindex}%
%  \usebibmacro{begentry}%
%  \usebibmacro{heading}%
%  \newunit
%  \usebibmacro{author}%
%  \setunit*{\labelnamepunct}%
%  \usebibmacro{thesistitle}%
%  \setunit{\respdelim}%
%  %\printnames[last-first:full]{author}%Вот эту строчку нужно убрать, чтобы автор диссертации не дублировался
%  \newunit\newblock
%  \printlist[semicolondelim]{specdata}%
%  \newunit
%  \usebibmacro{institution+location+date}%
%  \newunit\newblock
%  \usebibmacro{chapter+pages}%
%  \newunit
%  \printfield{pagetotal}%
%  \newunit\newblock
%  \usebibmacro{doi+eprint+url+note}%
%  \newunit\newblock
%  \usebibmacro{addendum+pubstate}%
%  \setunit{\bibpagerefpunct}\newblock
%  \usebibmacro{pageref}%
%  \newunit\newblock
%  \usebibmacro{related:init}%
%  \usebibmacro{related}%
%  \usebibmacro{finentry}}
%}{}

%\newbibmacro{string+doi}[1]{% новая макрокоманда на простановку ссылки на doi
%    \iffieldundef{doi}{#1}{\href{http://dx.doi.org/\thefield{doi}}{#1}}}

%\ifnumequal{\value{draft}}{0}{% Чистовик
%\renewcommand*{\mkgostheading}[1]{\usebibmacro{string+doi}{#1}} % ссылка на doi с авторов. стоящих впереди записи
%\renewcommand*{\mkgostheading}[1]{#1} % только лишь убираем курсив с авторов
%}{}
%\DeclareFieldFormat{title}{\usebibmacro{string+doi}{#1}} % ссылка на doi с названия работы
%\DeclareFieldFormat{journaltitle}{\usebibmacro{string+doi}{#1}} % ссылка на doi с названия журнала
%%% Тире как разделитель в библиографии традиционной руской длины:
\renewcommand*{\newblockpunct}{\addperiod\addnbspace\cyrdash\space\bibsentence}
%%% Убрать тире из разделителей элементов в библиографии:
%\renewcommand*{\newblockpunct}{%
%    \addperiod\space\bibsentence}%block punct.,\bibsentence is for vol,etc.

%%% Возвращаем запись «Режим доступа» %%%
%\DefineBibliographyStrings{english}{%
%    urlfrom = {Mode of access}
%}
%\DeclareFieldFormat{url}{\bibstring{urlfrom}\addcolon\space\url{#1}}

%%% В списке литературы обозначение одной буквой диапазона страниц англоязычного источника %%%
\DefineBibliographyStrings{english}{%
    pages = {p\adddot} %заглавность буквы затем по месту определяется работой самого biblatex
}

%%% В ссылке на источник в основном тексте с указанием конкретной страницы обозначение одной большой буквой %%%
%\DefineBibliographyStrings{russian}{%
%    page = {C\adddot}
%}

%%% Исправление длины тире в диапазонах %%%
% \cyrdash --- тире «русской» длины, \textendash --- en-dash
\DefineBibliographyExtras{russian}{%
  \protected\def\bibrangedash{%
    \cyrdash\penalty\value{abbrvpenalty}}% almost unbreakable dash
  \protected\def\bibdaterangesep{\bibrangedash}%тире для дат
}
\DefineBibliographyExtras{english}{%
  \protected\def\bibrangedash{%
    \cyrdash\penalty\value{abbrvpenalty}}% almost unbreakable dash
  \protected\def\bibdaterangesep{\bibrangedash}%тире для дат
}

%Set higher penalty for breaking in number, dates and pages ranges
\setcounter{abbrvpenalty}{10000} % default is \hyphenpenalty which is 12

%Set higher penalty for breaking in names
\setcounter{highnamepenalty}{10000} % If you prefer the traditional BibTeX behavior (no linebreaks at highnamepenalty breakpoints), set it to ‘infinite’ (10 000 or higher).
\setcounter{lownamepenalty}{10000}

%%% Set low penalties for breaks at uppercase letters and lowercase letters
%\setcounter{biburllcpenalty}{500} %управляет разрывами ссылок после маленьких букв RTFM biburllcpenalty
%\setcounter{biburlucpenalty}{3000} %управляет разрывами ссылок после больших букв, RTFM biburlucpenalty

%%% Список литературы с красной строки (без висячего отступа) %%%
%\defbibenvironment{bibliography} % переопределяем окружение библиографии из gost-numeric.bbx пакета biblatex-gost
%  {\list
%     {\printtext[labelnumberwidth]{%
%       \printfield{prefixnumber}%
%       \printfield{labelnumber}}}
%     {%
%      \setlength{\labelwidth}{\labelnumberwidth}%
%      \setlength{\leftmargin}{0pt}% default is \labelwidth
%      \setlength{\labelsep}{\widthof{\ }}% Управляет длиной отступа после точки % default is \biblabelsep
%      \setlength{\itemsep}{\bibitemsep}% Управление дополнительным вертикальным разрывом между записями. \bibitemsep по умолчанию соответствует \itemsep списков в документе.
%      \setlength{\itemindent}{\bibhang}% Пользуемся тем, что \bibhang по умолчанию принимает значение \parindent (абзацного отступа), который переназначен в styles.tex
%      \addtolength{\itemindent}{\labelwidth}% Сдвигаем правее на величину номера с точкой
%      \addtolength{\itemindent}{\labelsep}% Сдвигаем ещё правее на отступ после точки
%      \setlength{\parsep}{\bibparsep}%
%     }%
%      \renewcommand*{\makelabel}[1]{\hss##1}%
%  }
%  {\endlist}
%  {\item}

%%% Макросы автоматического подсчёта количества авторских публикаций.
% Печатают невидимую (пустую) библиографию, считая количество источников.
% http://tex.stackexchange.com/a/66851/79756
%
\makeatletter
    \newtotcounter{citenum}
    \defbibenvironment{counter}
        {\setcounter{citenum}{0}\renewcommand{\blx@driver}[1]{}} % begin code: убирает весь выводимый текст
        {} % end code
        {\stepcounter{citenum}} % item code: cчитает "печатаемые в библиографию" источники

    \newtotcounter{citeauthorvak}
    \defbibenvironment{countauthorvak}
        {\setcounter{citeauthorvak}{0}\renewcommand{\blx@driver}[1]{}}
        {}
        {\stepcounter{citeauthorvak}}

    \newtotcounter{citeauthorscopus}
    \defbibenvironment{countauthorscopus}
        {\setcounter{citeauthorscopus}{0}\renewcommand{\blx@driver}[1]{}}
        {}
        {\stepcounter{citeauthorscopus}}

    \newtotcounter{citeauthorwos}
    \defbibenvironment{countauthorwos}
        {\setcounter{citeauthorwos}{0}\renewcommand{\blx@driver}[1]{}}
        {}
        {\stepcounter{citeauthorwos}}

    \newtotcounter{citeauthorother}
    \defbibenvironment{countauthorother}
        {\setcounter{citeauthorother}{0}\renewcommand{\blx@driver}[1]{}}
        {}
        {\stepcounter{citeauthorother}}

    \newtotcounter{citeauthorconf}
    \defbibenvironment{countauthorconf}
        {\setcounter{citeauthorconf}{0}\renewcommand{\blx@driver}[1]{}}
        {}
        {\stepcounter{citeauthorconf}}

    \newtotcounter{citeauthor}
    \defbibenvironment{countauthor}
        {\setcounter{citeauthor}{0}\renewcommand{\blx@driver}[1]{}}
        {}
        {\stepcounter{citeauthor}}

    \newtotcounter{citeauthorvakscopuswos}
    \defbibenvironment{countauthorvakscopuswos}
        {\setcounter{citeauthorvakscopuswos}{0}\renewcommand{\blx@driver}[1]{}}
        {}
        {\stepcounter{citeauthorvakscopuswos}}

    \newtotcounter{citeauthorscopuswos}
    \defbibenvironment{countauthorscopuswos}
        {\setcounter{citeauthorscopuswos}{0}\renewcommand{\blx@driver}[1]{}}
        {}
        {\stepcounter{citeauthorscopuswos}}

    \newtotcounter{citeregistered}
    \defbibenvironment{countregistered}
        {\setcounter{citeregistered}{0}\renewcommand{\blx@driver}[1]{}}
        {}
        {\stepcounter{citeregistered}}

    \newtotcounter{citeauthorpatent}
    \defbibenvironment{countauthorpatent}
        {\setcounter{citeauthorpatent}{0}\renewcommand{\blx@driver}[1]{}}
        {}
        {\stepcounter{citeauthorpatent}}

    \newtotcounter{citeauthorprogram}
    \defbibenvironment{countauthorprogram}
        {\setcounter{citeauthorprogram}{0}\renewcommand{\blx@driver}[1]{}}
        {}
        {\stepcounter{citeauthorprogram}}

    \newtotcounter{citeexternal}
    \defbibenvironment{countexternal}
        {\setcounter{citeexternal}{0}\renewcommand{\blx@driver}[1]{}}
        {}
        {\stepcounter{citeexternal}}
\makeatother

\defbibheading{nobibheading}{} % пустой заголовок, для подсчёта публикаций с помощью невидимой библиографии
\defbibheading{pubgroup}{\section*{#1}} % обычный стиль, заголовок-секция
\defbibheading{pubsubgroup}{\noindent\textbf{#1}} % для подразделов "по типу источника"

%%%Сортировка списка литературы Русский-Английский (предварительно удалить dissertation.bbl) (начало)
%%%Источник: https://github.com/odomanov/biblatex-gost/wiki/%D0%9A%D0%B0%D0%BA-%D1%81%D0%B4%D0%B5%D0%BB%D0%B0%D1%82%D1%8C,-%D1%87%D1%82%D0%BE%D0%B1%D1%8B-%D1%80%D1%83%D1%81%D1%81%D0%BA%D0%BE%D1%8F%D0%B7%D1%8B%D1%87%D0%BD%D1%8B%D0%B5-%D0%B8%D1%81%D1%82%D0%BE%D1%87%D0%BD%D0%B8%D0%BA%D0%B8-%D0%BF%D1%80%D0%B5%D0%B4%D1%88%D0%B5%D1%81%D1%82%D0%B2%D0%BE%D0%B2%D0%B0%D0%BB%D0%B8-%D0%BE%D1%81%D1%82%D0%B0%D0%BB%D1%8C%D0%BD%D1%8B%D0%BC
%\DeclareSourcemap{
%    \maps[datatype=bibtex]{
%        \map{
%            \step[fieldset=langid, fieldvalue={tempruorder}]
%        }
%        \map[overwrite]{
%            \step[fieldsource=langid, match=russian, final]
%            \step[fieldsource=presort,
%            match=\regexp{(.+)},
%            replace=\regexp{aa$1}]
%        }
%        \map{
%            \step[fieldsource=langid, match=russian, final]
%            \step[fieldset=presort, fieldvalue={az}]
%        }
%        \map[overwrite]{
%            \step[fieldsource=langid, notmatch=russian, final]
%            \step[fieldsource=presort,
%            match=\regexp{(.+)},
%            replace=\regexp{za$1}]
%        }
%        \map{
%            \step[fieldsource=langid, notmatch=russian, final]
%            \step[fieldset=presort, fieldvalue={zz}]
%        }
%        \map{
%            \step[fieldsource=langid, match={tempruorder}, final]
%            \step[fieldset=langid, null]
%        }
%    }
%}
%Сортировка списка литературы (конец)

%%% Создание команд для вывода списка литературы %%%
\newcommand*{\insertbibliofull}{
    \printbibliography[keyword=bibliofull,section=0,title=\bibtitlefull]
    \ifnumequal{\value{draft}}{0}{
      \printbibliography[heading=nobibheading,env=counter,keyword=bibliofull,section=0]
    }{}
}
\newcommand*{\insertbiblioauthor}{
    \printbibliography[heading=pubgroup, section=0, filter=papersregistered, title=\bibtitleauthor]
}
\newcommand*{\insertbiblioauthorimportant}{
    \printbibliography[heading=pubgroup, section=2, filter=papersregistered, title=\bibtitleauthorimportant]
}

% Вариант вывода печатных работ автора, с группировкой по типу источника.
% Порядок команд `\printbibliography` должен соответствовать порядку в файле common/characteristic.tex
\newcommand*{\insertbiblioauthorgrouped}{
    \section*{\bibtitleauthor}
    \ifsynopsis
    \printbibliography[heading=pubsubgroup, section=0, keyword=biblioauthorvak,    title=\bibtitleauthorvak,resetnumbers=true] % Работы автора из списка ВАК (сброс нумерации)
    \else
    \printbibliography[heading=pubsubgroup, section=0, keyword=biblioauthorvak,    title=\bibtitleauthorvak,resetnumbers=false] % Работы автора из списка ВАК (сквозная нумерация)
    \fi
    \printbibliography[heading=pubsubgroup, section=0, keyword=biblioauthorwos,    title=\bibtitleauthorwos,resetnumbers=false]% Работы автора, индексируемые Web of Science
    \printbibliography[heading=pubsubgroup, section=0, keyword=biblioauthorscopus, title=\bibtitleauthorscopus,resetnumbers=false]% Работы автора, индексируемые Scopus
    \printbibliography[heading=pubsubgroup, section=0, keyword=biblioauthorpatent, title=\bibtitleauthorpatent,resetnumbers=false]% Патенты
    \printbibliography[heading=pubsubgroup, section=0, keyword=biblioauthorprogram,title=\bibtitleauthorprogram,resetnumbers=false]% Программы для ЭВМ
    \printbibliography[heading=pubsubgroup, section=0, keyword=biblioauthorconf,   title=\bibtitleauthorconf,resetnumbers=false]% Тезисы конференций
    \printbibliography[heading=pubsubgroup, section=0, keyword=biblioauthorother,  title=\bibtitleauthorother,resetnumbers=false]% Прочие работы автора
}

\newcommand*{\insertbiblioexternal}{
    \printbibliography[heading=pubgroup,    section=0, keyword=biblioexternal,     title=\bibtitlefull]
}
     % Реализация пакетом biblatex через движок biber
}

% Вывести информацию о выбранных опциях в лог сборки
\typeout{Selected options:}
\typeout{Draft mode: \arabic{draft}}
\typeout{Font: \arabic{fontfamily}}
\typeout{AltFont: \arabic{usealtfont}}
\typeout{Bibliography backend: \arabic{bibliosel}}
\typeout{Precompile images: \arabic{imgprecompile}}
% Вывести информацию о версиях используемых библиотек в лог сборки
\listfiles

%%% Управление компиляцией отдельных частей диссертации %%%
% Необходимо сначала иметь полностью скомпилированный документ, чтобы все
% промежуточные файлы были в наличии
% Затем, для вывода отдельных частей можно воспользоваться командой \includeonly
% Ниже примеры использования команды:
%
%\includeonly{Dissertation/part2}
%\includeonly{Dissertation/contents,Dissertation/appendix,Dissertation/conclusion}
%
% Если все команды закомментированы, то документ будет выведен в PDF файл полностью

\begin{document}
%%% Переопределение именований типовых разделов
% https://tex.stackexchange.com/a/156050
\gappto\captionsrussian{\input{common/renames}\unskip} % for polyglossia and babel
\input{common/renames}

%%% Структура диссертации (ГОСТ Р 7.0.11-2011, 4)
\include{Dissertation/title}           % Титульный лист
% \textit{Abstract}

Automated testing frameworks are widely used for assuring quality of modern software in secure software development lifecycle. Sometimes it is needed to assure quality of specific software and, hence specific approach should be applied. In this work an approach and implementation details of automated testing framework suitable for acceptance testing of static source code analysis tools is presented. The presented framework is used for continuous testing of static source code analyzers for C, C++ and Python programs.
     % Аннотация анг.
% \textit{Аннотация}

Среды автоматизированного тестирования широко используются для обеспечения качества современного программного обеспечения в безопасном жизненном цикле разработки программного обеспечения. Иногда необходимо обеспечить качество конкретного программного обеспечения и, следовательно, следует применять особый подход. В этой статье мы представляем подход и детали реализации автоматизированной среды тестирования, подходящей для приемочного тестирования инструментов статического анализа исходного кода. Представленный фреймворк используется для непрерывного тестирования статических анализаторов исходного кода программ на языках C, C ++ и Python.
     % Аннотация рус.
\include{Dissertation/contents}        % Оглавление
\ifnumequal{\value{contnumfig}}{1}{}{\counterwithout{figure}{chapter}}
\ifnumequal{\value{contnumtab}}{1}{}{\counterwithout{table}{chapter}}
\include{Dissertation/introduction}    % Введение
\ifnumequal{\value{contnumfig}}{1}{\counterwithout{figure}{chapter}
}{\counterwithin{figure}{chapter}}
\ifnumequal{\value{contnumtab}}{1}{\counterwithout{table}{chapter}
}{\counterwithin{table}{chapter}}
\chapter{Обзор}\label{ch:ch1}

\section{История автоматического анализа программ}\label{sec:ch1/sec1}
% Ранняя история верификации программ
%TODO: add references
В 1947 году появились термины «ошибка» (bug) и «отладка» (debugging). Грейс Мюррей, 
ученая из Гарвардского университета, работавшая с компьютером Mark II, обнаружила, 
что мотылек застрял в реле, из-за чего оно не вступало в контакт. Она подробно 
описала инцидент в рабочем журнале, приклеив мотылька лентой в качестве доказательства и 
назвав мотылька «ошибкой», вызывающей ошибку, а действие по устранению ошибки - «отладкой».
%TODO: reformulate bug

В то время тесты были сосредоточены на оборудовании, потому что оно было не так развито, как 
сегодня, и его надежность была важна для правильного функционирования программного обеспечения.
Термин отладка был связан с применением исправлений для конкретной ошибки как одна из фаз в 
стадии разработки программного обеспечения. Проводимые тесты имели коррекционный характер и 
выполнялись для устранения ошибок, не дававших программе работать. 

В 1957 году Чарльз Бейкер объясняет необходимость разработки тестов, чтобы гарантировать, что 
программное обеспечение соответствует заранее разработанным требованиям (тестирование), а также 
функциональным возможностям программы (отладка). Разработка тестов стала более важной по мере 
того, как разрабатывались более дорогие и сложные приложения, и стоимость устранения всех этих 
недостатков оказывала явный риск для прибыльности проекта. Особое внимание было уделено 
увеличению количества и качества тестов, и впервые качество продукта стало связано с фазой 
тестирования. Цель заключалась в том, чтобы продемонстрировать, что программа выполняет то, что 
от нее требовалось, с использованием ожидаемых и выдаваемых параметров.

В 1979 году Гленфорд Дж. Майерс радикально меняет процедуру обнаружения ошибок в программе:
"Тестирование программного обеспечения - это процесс запуска программы с целью поиска ошибок."
Обеспокоенность Майерса заключалась в том, что, преследуя цель продемонстрировать, что программа 
безупречна, можно подсознательно выбрать тестовые данные, которые имеют низкую вероятность 
вызвать сбои программы, тогда как если цель состоит в том, чтобы продемонстрировать, что 
программа ошибочна, тестовые данные будут имеют большую вероятность их обнаружения, и мы будем 
более успешными в тестировании и, следовательно, в качестве программного обеспечения. C этого
момента тесты будут пытаться продемонстрировать, что программа не работает должным образом, в 
отличие от того, как это делалось ранее, что приведет к новым методам тестирования и анализа.

В 1983 году была предложена методология, которая объединяет действия по анализу, пересмотру 
(revision) и тестированию в течение жизненного цикла программного обеспечения, чтобы получить 
оценку продукта в процессе разработки. Этап тестирования признан неотъемлемым этапом в 
разработке продукта, приобретая особое значение в связи с появлением инструментов для разработки 
автоматизированных тестов, которые заметно повысили эффективность.

В 1988 году Уильям Хетцель опубликовал «Рост тестирования программного обеспечения», в котором 
он переопределил концепцию тестирования как планирование, проектирование, создание, обслуживание 
и выполнение тестов и тестовых сред. Это  в основном отразилось на появлении фазы тестирования 
на самом раннем этапе разработки продукта, этапе планирования. Если мы представим весь процесс 
разработки в виде конечной линии, где начало - это планирование, а конец - мониторинг проданного 
продукта, мы увидим, как фаза тестирования переместилась влево. Онa появилась как этап пост-
продакшнен, позже это был этап предпродакшн, а сейчас она находится на стадии завершения. Эта 
практика известна как Shift-Left.

Э. В. Дейкстра в лекции «О надёжности программ» утверждает, что тесты могут показать наличие 
ошибок в программе, но не могут доказать их отсутствие\autocite{Dijkstra}. Таким образом 
использование одного тестирования в процессе верификации программы не является достаточным. 

\subsection{Уровни тестирования ПО}
Процесс тестирования представляет из себя несколько уровней. 

Модульное тестирование (Unit testing) - это процесс тестирования отдельных подпрограмм, или 
процедур в программе. Прежде чем тестировать всю программу целиком следует сконцетрироваться 
на отдельных ее частях. Это объясняется рядом причин. Во первых, модульное тестирование облегчает задачу
отладки(процесс выявления и исправления обнаруженной ошибки), так как, когда ошибка найдена 
область ее распространения ограничена размерами модуля. Во вторых, это позволяет распараллелить 
процесс тестировая проверяя сразу несколько модулей одновременно. Целью модульного тестирования
является сравнение значения функции модуля некоторой функции или интерфейсу спецификации 
опредленной модулем. Следует подчеркнуть, что цель, как и для любого процесса тестирования, в том
чтобы найти несоответствие спецификации. 

Когда заканчивается модульное тестирование программы, в действительности процесс тестирования 
только начинается. Особенно это касается больших или сложных программ. Программная ошибка возникает, когда программа не выполняет то, что ее конечный пользователь обоснованно ожидает от 
нее. Согласно данному определения, даже проведя абсолютно идеальный модульный тест нельзя 
утверждать, что все ошибки найдены. Таким образом, для завершения тестирования необходимо какое-
то дополнительное тестирование. В\autocite{artoftesting} это называется тестированием высшего 
порядка.

Разработка программного обеспечения - это в значительной степени процесс передачи информации о 
реализуемой программе и перевода этой информации из одной формы в другую. По этой причине 
подавляющее большинство ошибок программного обеспечения можно отнести к сбоям, ошибкам и 
помехам во время передачи и перевода информации.

Ход процесса разработки ПО можно описать семью шагами:

1. Требования пользователя программы переводятся в набор письменных требований. Это цели продукта.
2. Требования преобразуются в конкретные цели путем оценки осуществимости и стоимости, разрешения противоречивых требований и установления приоритетов и компромиссов.
3. Цели переводятся в точную спецификацию продукта, в которой продукт рассматривается как черный ящик и учитываются только его интерфейсы и взаимодействие с конечным пользователем. Это описание называется внешней спецификацией.
4. Если продукт представляет собой систему, такую как операционная система, система управления полетом, система управления базами данных или кадровая система сотрудников, а не программа (компилятор, программа расчета заработной платы, текстовый процессор), следующим процессом является проектирование системы. На этом этапе система разделяется на отдельные программы, компоненты или подсистемы и определяется их интерфейсы.
5. Структура программы или программ разрабатывается путем определения функции каждого модуля, иерархической структуры модулей и интерфейсов между модулями.
6. Разработана точная спецификация, определяющая интерфейс и функции каждого модуля.
7. Через один или несколько подшагов спецификация интерфейса модуля транслируется в алгоритм исходного кода каждого модуля.

Учитывая предпосылку, что семь этапов цикла разработки включают в себя общение, понимание и 
перевод информации, а также предпосылку, что большинство ошибок программного обеспечения 
происходит из-за сбоев в обработке информации, существует три дополнительных подхода для 
предотвращения и/или обнаружения этих ошибок. Во-первых, мы можем внести больше точности в 
процесс разработки, чтобы предотвратить многие ошибки. Во-вторых, мы можем ввести в конце 
каждого процесса отдельный этап проверки, чтобы определить как можно больше ошибок, прежде чем 
переходить к следующему процессу. Третий подход - ориентировать отдельные процессы 
тестирования на отдельные процессы разработки. То есть сосредоточить каждый процесс 
тестирования на конкретном этапе перевода, таким образом фокусируя его на определенном классе ошибок. %TODO: Add image.
Другими словами, вы должны иметь возможность установить взаимно однозначное соответствие между процессами разработки и тестирования.

Функциональное тестирование (Functional testing) - это попытка найти расхождения между 
программой и внешней спецификацией. Внешняя спецификация - это точное описание поведения 
программы с точки зрения конечного пользователя. За исключением случаев использования в 
небольших программах, функциональное тестирование обычно представляет собой черный ящик.
Чтобы выполнить функциональный тест, спецификация анализируется для получения набора тестовых 
примеров.

Системное тестирование (System testing) - это самый непонятый и самый сложный из процессов 
тестирования. Системное тестирование - это не процесс тестирования функций всей системы или 
программы, потому что это будет дублировать процесс функционального тестирования. Системное 
тестирование имеет конкретную цель: сравнить систему или программу с ее первоначальными 
целями.

Приемочное тестирование (Acceptance testing) - это процесс сравнения программы с исходными требованиями 
и текущими потребностями конечных пользователей. Это необычный тип тестирования, поскольку он обычно 
выполняется заказчиком программы или конечным пользователем и обычно не считается обязанностью 
организации-разработчика.

На ряду с тестированием одним из подходов к написанию безопасного и переносимого ПО 
является использование в разработке стандартов кодирования. 

\subsection{Правила кодирования}

Спустя полвека с момента своего создания язык программирования C по-прежнему остается одним из наиболее часто 
используемых языков программирования\autocite{TiobeIndex} и наиболее часто используемый для разработки 
встраиваемых систем. Причины такого успеха уходят корнями в требования отрасли. Среди таких требований - размер 
языка, стабильность и путь развития, обеспечивающий обратную совместимость.

Характеристики, которые сделали язык программирования C таким успешным, имеют недостатки: написание безопасных 
и защищенных приложений на C требует особой осторожности. В противном случае код может получиться 
запутанным и неясным. Замечательные примеры подобных программ представлены на международном конкурсе 
запутанного С кода \autocite{ioccc}. Решение, обязательное или настоятельно рекомендуемое 
всеми применимыми промышленными стандартами - language subsetting: критически важные приложения не 
программируются на неограниченном языке C, а программируются в подмножестве, где вероятность совершения 
потенциально опасных ошибок снижена. Это требуется или настоятельно рекомендуется всеми стандартами 
безопасности, таким как, например, IEC 61508\autocite{IEC}.

Конечно, кодирования на более безопасном подмножестве C недостаточно для гарантии корректности. Однако:

\begin{itemize}
    \item Ограничение языкового подмножества, в котором не полностью определенное поведение и 
    проблемные функции запрещены или строго регулируются, «может значительно повысить эффективность и  
    точность статического анализа»\autocite{astreeConf};
    \item Правильно спроектированные языковые подмножества имеют сильное влияние на читаемость кода:
    ревью кода в сочетании со статическим анализом и соблюдением правил кодирования являются основой 
    наиболее эффективных стратегий устранения дефектов.  
\end{itemize}

Согласно\autocite{bagnara2020barrc2018} в опросе среди специалистов втсроенных систем ПО  стандарт кодирования MISRA C\autocite{Misrac1998} - наиболее широко 
используемый и стандарт BARR-C\autocite{barrc} - второй наиболее используемый. Вместе они получили 40 процентов голосов респондентов.  

Проект MISRA (Motor Industry Software Reliability Association) был основан для создания руководства по разработке ПО для микроконтроллеров в 
наземных средствах по заказу правительства Британии. Работа над проетом началась в 1990 году.  Первое руководство вышло в 1994 году, не 
было привязано к какому-либо языку. Консорциум MISRA приступил к работе на стандартом для языка C: в это же время Форд и Ленд Ровер независимо разрабатывали проприетарные 
руководства для программного обеспечения на языке C для транспортных средств, и было признано, что совместная 
деятельность будет более выгодной для промышленности. Первый связанный с языком С стандрат MISRA C\autocite{Misrac1998} вышел в 1998 году и стал общепринятым. 

В 2004 году MISRA опубликовало улучшенную версию стандарта\autocite{Misrac2004}, расширев целевую аудиторию, 
чтобы включить все отрасли, которые разрабатывают программное обеспечение на C для использования в 
высоконадежных/критических системах. Благодаря успеху MISRA C и тому факту, что C++ также используется в 
критических контекстах, в 2008 г. MISRA опубликовала аналогичный набор рекомендаций MISRA C++
\autocite{Misrac2008}.

И MISRA C:1998, и MISRA C:2004 нацелены на версию стандарта C 1990 г.\autocite{C-ISOIEC99899-1990}. Последняя 
версия MISRA C:2012, опубликованная в 2013 г.\autocite{Misrac2012} поддерживает оба стандарта: С90 и 
С99\autocite{C-ISOIEC99899-1999}. По сравнению с предыдущими версиями, MISRA C:2012 охватывает больше языковых 
проблем и предоставляет более точную спецификацию с улучшенными обоснованием и примерами.

MISRA C повлиял на все общедоступные стандарты кодирования для C и C++, которые были разработаны после MISRA 
C:1998. MISRA C:1998 повлиял на JSF Air Vehicle C++ Coding Standards\autocite{JSF}, который в свою очередь повлиял на MISRA C++:2008.

В MISRA C правила деляться на три основных категории: Mandatory, Required и Advisory. Mandatory -
наиболее строгая категория, требующая постоянного выполнения. Required - менее строгая: возможны 
отклонения при условии документирования и обоснования. Advisory - правила, которым следовать 
не обязательно.

В MISRA-C:1998 перечислено 127 правил (93 обязательных и 34 рекомендательных).

В MISRA-C:2004 141 правило (121 обязательное и 20 рекомендательных). Правила разделены на 21 категорию.

В MISRA-C:2012 143 правила (каждое из которых может быть проверено статическим анализатором кода) и 16 директив 
(правил, соответствие которым открыто для интерпретаций или связано с процессами и процедурами). Правила делятся 
на обязательные, требуемые и рекомендательные; могут распространяться на отдельные единицы трансляции или на всю 
систему.

На стандарт BARR:2018 оказали влияние MISRA C:2012, BARR C:2013 и NETRINO EMBEDDED C. Как следстсвие BARR:2018 является 
надмножеством MISRA C:2012 и может использоваться, как плавный первый шаг для проекта без статического анализа и правил кодирования 
к проекту, соответствующиму MISRA C:2012.

Другими примероми стандарта кодирования для обеспечения безопасности критических систем являются:
\begin{itemize}
    \item "JPL Institutional Coding Standard for the C Programming Language"\autocite{JPL} от НАСА. Данный стандарт испытал влияние MISRA C:2004 и 
        стандарта "Сила десяти"\autocite{powerOfTen}. Ни один из этих двух данных источников не рассматривает программные риски, связанные с использованием 
        многопоточного программного обеспечения. Этот стандарт призван заполнить этот пробел.
    \item "High Integrity C++"\autocite{highIntegrity} от Programming Research Ltd. 
    \item "JOINT STRIKE FIGHTER AIR VEHICLE C++ CODING STANDARDS"\autocite{JSF} от Lockheed.
    \item "Embedded System development Coding Reference guide for C/C++"\autocite{escrC, escrCPP} от Information-technology Promotion Agency, Japan.
    \item "Guidelines for the use of the C++14 language in critical and safety-related systems"\autocite{autosar} от AUTOSAR.
    \item "The SEI CERT C/C++ Coding Standart" от Carnegie Mellon University\autocite{CERT}. 
\end{itemize}

\subsection{Бенчмарки для проверки статических анализаторов}
Бенчмаркинг обеспечивает объективный и повторяемый способ измерения свойств инструмента обнаружения ошибок. Бенчмарки для инструментов обнаружения 
ошибок должен отвечать на несколько вопросов о результатах работы инструмента\autocite{}: 
- Сколько ошибок обнаружены правильно?
- Сколько ложных сообщений об ошибках сделано?
- Сколько ошибок пропущено/не сообщается?
- Насколько хорошо масштабируется инструмент?

Не на все эти вопросы можно ответить с помощью одного тестового набора. Чтобы измерить масштабируемость инструмента, необходимы большие распределения 
кода с миллионами строк кода, которые будут репрезентативными для реального мира. Однако определение того, сколько ошибок сообщается правильно и 
неправильно или сколько ошибок пропущено, является невозможно для больших баз кода из-за практических ограничения на обнаружение всех ошибок в большой 
программе. Таким образом выделяются два основных типа бенчмарков. 

Маленькие бенчмарки размером от 10 до тысячи строк кода состоят из синтетических тестов, либо из автономных программы, извлеченные из существующего 
ошибочного кода, которые содержат конкретную ошибку. В работе\autocite{Zitser} извлечено 15 багов из реальных приложений. В работе\autocite{Kratkiewicz} синтезировано 
291 тесткейса ошибки переполнения буфера. Один из наиболее известных для данных целей тестовый набор - Juliet Test Suite\autocite{NIST}. Juliet Test Suite был разработан Центр гарантированного 
программного обеспечения (Center for Assured Software) Агентства национальной безопасности США (US American National Security Agency).
Его тестовые сценарии были созданы для тестирования сканеров или другого программного обеспечения. Набор тестов состоит из двух частей. Одна часть 
посвящена ошибкам безопасности для языков программирования C и C++. Другой касается ошибок безопасности для языка Java. Примеры кода с уязвимостями 
безопасности даны в простой форме, а также встроены в вариации различных потоков управления и паттернов потоков данных. Пакет содержит около 57 000 
тестовых случаев на C/C++ и около 24000 тестовых случаев на Java. Набор тестов охватывает 25 основных ошибок безопасности, определенных SANS/MITRE 
(MITRE 2011)\autocite{MITRE}. Можно выделить два типа исходного кода: искусственный код и естественный код. Естественный код используется в реальном программном обеспечении, таком 
как, например, веб-сервер Apache или Microsoft Word. Искусственный код создается для определенной цели, например, для тестирования сканеров 
безопасности. Juliet Test Suite содержит только искусственный код, поскольку такой код упрощает оценку и сравнение инструментов статического анализа. 

Примерами больших бенчмарков в виде полных программных дистрибутивов являются BugBench\autocite{BugBench} и Faultbench\autocite{Faultbench}.

Данные бенчмарки не являются общими, переносными или многоразовыми; все ключевые свойства, позволяющие сделать набор тестов полезным.  
BegBunch\autocite{BegBunch} устраняет эти недостатки, предоставляя два набора для оценки качественной и количественной производительности инструментов 
обнаружения ошибок, а также автоматизирует запуск  инструмента обнаружения ошибок, проверку результатов и составления отчетов о данных производительности. 

\subsection{Первое поколение статических анализаторов. Lint}

Первый инструмент статического анализа Lint появился в конце 1970-x. Впервые разработчики 
получили возможность автоматизировать обнаружение дефектов программного обеспечения на самых 
ранних этапах жизненного цикла приложения, когда их легче всего исправить. Кроме того, это 
давало разработчикам уверенность в качестве своего кода перед релизом. Технология, лежащая в 
основе Lint, была революционной, она использовала компиляторы для проверки дефектов. 

Однако Lint не разработатывался с целью выявления дефектов, вызывающих проблемы во 
время выполнения программы. Скорее, его цель заключалась в том, чтобы выделить подозрительные 
или непереносимые конструкции в коде и помочь разработчикам соблюдать общий формат при 
кодировании. Под "подозрительным кодом" имееся в виду код, который, будучи технически правильным 
с точки зрения языка исходного кода (например C, С++), может быть структурирован так, чтобы он 
выполнялся способами, которые разработчик не предполагал. Lint являлся дополнением к компилятору. В то время как  компилятор концетрировался на быстром и точном превращении программы в послдовательность бит, Lint концетрировался на ошибках в переносимости, стиле и эффективности.
Из-за ограниченных возможностей анализа Lint, уровень шума был чрезвычайно высоким, часто превышая соотношение между шумом и реальными дефектами в соотношении 10 к 1.

Следовательно, обнаружение настоящих дефектов требовало от разработчиков проведения трудоемкой 
ручной проверки результатов Lint, что усложняло именно ту проблему, которую должен был 
устранить статический анализ. По этой причине Lint так и не получил широкого распространения в 
качестве инструмента обнаружения дефектов, хотя имел ограниченный успех в нескольких 
организациях. Фактически, как свидетельство качества технологии, лежащей в основе Lint, 
множество различных версий продукта по-прежнему доступны сегодня. %TODO: Add links to products

\subsection{Второе поколение статических анализаторов}
Почти два десятилетия статический анализ оставался скорее фикцией, чем коммерчески 
жизнеспособным производственным инструментом для выявления дефектов. В начале 2000 года
появилось второе поколение инструментов (Stanford Checker). Используя новые 
технологии, оно расширяло возможности инструментов первого поколения от простого выявления 
нежелательных паттернов до покрытия путей выполнения. Эти инструменты могли анализировать 
целые базы кода, а не только один файл. 

Сместив фокус с "подозрительных конструкций" на "дефекты времени выполнения", разработчики 
статического анализа осознали необходимость в большем понимания внутреннего 
устройста программ. Это означало объединение сложного анализа путей с межпроцедурным анализом, 
чтобы понять, что происходит, когда поток управления переходит от одной функции к другой в 
рамках данной программы.

Несмотря на принятие и использование организациями, статический анализ 2-го поколения все еще 
не мог найти золотую середину между точностью и масштабируемостью. Некоторые решения были 
точными для небольшого набора типов дефектов, но не могли масштабироваться для анализа 
миллионов строк кода. Другие могли работать за короткое время, но имели показатели точности, 
аналогичные Lint, представляя похожие проблемы с ложными срабатываниеми и шумом. После 
внедрения в процесс разработки эти инструменты могут сообщать о дефектах в приемлемом 
соотношении, но только с ограниченными параметрами анализа. 

Инструментам второго поколения требовались однородные среды сборки и разработки. 
Из-за этого внедрить их в какой то проект было сложной задачей, требовавшей больших усилий.

Борьба между точностью и маштабируемостью вылилась в проблему с ложными срабатываниями. Подобно 
шуму, от которого страдали инструменты первого поколения, ложные срабатывания тормозили 
распространению инструментов нового поколения. 

\subsection{Третье поколение статических анализаторов}
%TODO: add date of birth for 3rd generation
Третье поколение инструментов статической проверки программ превосходит своих предшественников по
всем параметрам и является неотъемлемой частью процессов и сред разработки.  

\subsection{}


\FloatBarrier
           % Глава 1
\chapter{Требования}\label{ch:ch2}

\section{Выбор характеристик для оценки качества инструментов статического анализа}\label{sec:ch2/sec1}
Результат запуска статического анализатора на тестовом наборе или проекте это множество верных и ложных предупреждений об ошибках (соответственно
true positives -- TP, false positives -- FP). Ложное предупреждение об ошибке свидетельствует о наличие неточности в алгоритме статического анализатора. Большое
количество ложных предупреждений -- это шум в результатах работы статического анализатора и признак низкого качества проводимого анализа. Большое
количество верных предупреждений при малом количестве ложных предупреждений свидетельствует о высоком качестве проводимого анализа и о его полноте.
Под полнотой понимается способность статического анализатора обнаруживать
как можно больше дефектов, являющихся истинными ошибками в программе. Отношение верных предупреждений об ошибках
к общему количеству предупреждений это метрика -- точность (precision) \eqref{eq:precision}. 
\begin{equation}\label{eq:precision}
precision = TP/(TP + FP)
\end{equation}
Данная метрика используется в статистике для оценки результатов исследования на основе полученных true positives и false positives. Данная метрика широко распространена в
оценке качества статических анализаторов. 
Общее количество предупреждений
формируется из суммы верных и ложных предупреждений. Ситуации, в которых инструмент не выдал предупреждения, однако ошибка имеет место быть,
являются false negatives. Сумма false negatives и true positives это общее количество ошибок, которое содержится в тестовом наборе или коде проекта. Чаще
эта величина вычисляется для статических анализаторов, сканирующих тестовые наборы, так как определение точного количества ошибок в программном
проекте –- сложная задача. Отношение количества верных предупреждений, true positives, к
общему количеству ошибок в тестовом наборе или проекте это мера оценки качества результата инструмента статического анализа, называемая отклик (recall) \eqref{eq:recall}. 
\begin{equation}\label{eq:recall}
recall = TP/(TP+FN)
\end{equation}
Данная мера так же как и точность используется в статистике для оценки качества проводимого
исследования. Идеальный инструмент статического анализа не выдает ложных
предупреждений и сообщает о всех существующих ошибках. Для идеального инструмента значения точности и отклика равны единице. Теоретически полностью
противоположный идеальному инструменту инструмент выдает только ложные
предупреждения, шум. Для такого инструмента значения точности и отклика равны
нолю. Таким образом значения точности и отклика находятся в диапазоне от ноля до
единицы. На основе значений точности и отклика вычисляется величина F-measure.
F-measure характеризует точность проводимого исследования. 
F-measure вычисляется, как гармоническое среднее точности и отклика \eqref{eq:fmeasure}. 
\begin{equation}\label{eq:fmeasure}
F-measure = 2 * precision * recall / (precision + recall)
\end{equation}
Наибольшее возможное значение для F-measure равно одному, соответсвующее идеальным значениям точности и отклика.
Наименьшее возможное значение для F-measure равно нолю, в данном случае одно из величин, точность или отклик равно нолю.

Чтобы измерить величины точность, отклик, F-measure для участвующего в исследовании статического анализатора
предлагается следующий порядок действий. Во-первых, сформировать тестовый набор. Данный тестовый набор должен соответствовать инспекциям (чекерам) статического
анализатора, ошибкам которые он обнаруживает. Во-вторых, запустить статический анализатор на данном тестовом наборе. Данный шаг осуществляется
средствами Acceptance Testing Framework, или сокращенно ATF. В-третьих, собрать статистику работы статического анализатора
на тестовом наборе. Данный шаг осуществляется средствами ATF. В-четвертых,
на основе полученной статистики, которая есть количество верных и ложных
предупреждений, вычислить значения точности и отклика. 

Результаты измерений будут
представлены в двух интерпретациях. Первая интерпретация -- значения величин
точность и отклик для тестового набора, на котором проводился анализ для данного статического анализатора, то есть проверяется качество результата инструмента в рамках заявленных возможностей обнаружения определённого типа ошибок. Вторая интерпретация –- значения величин 
точность и отклик на тестовых примерах, представляющих полное количество возможных обнаруживаемых дефектов в рамках тестового набора. Вторая интерпретация необходима для объективного сравнения качества
анализа, проводимого статическими анализаторами.

\section{Требования к Acceptance Testing Framework}\label{sec:ch2/sect2}
Инструменты статического анализа исходного кода должны проверять состояние исходного кода программ с точки зрения очень разных правил, которые
могут применяться в качестве промышленного или общекорпоративного стандарта кодирования. Несмотря на то, что современные статические анализаторы
исходного кода уделяют особое внимание безопасности кода, отсутствию логических ошибок и производительности, некоторые правила кодирования, применяемые в компаниях или отрасли, могут содержать такие требования к коду, как
стиль отступов, соглашения об именах и т. д. Например, если рассмотреть
программы, написанные на языке программирования Python, то исходный код может содержать комментарии определенного вида, такие как Shebang, кодировка файла, информация о версии или лицензии. 
Вот почему, пытаясь удовлетворить потребности тестирования промышленных статических анализаторов исходного кода, такой фреймворк не может полагаться на специальные комментарии и форматирование кода, как, например, используемые в большинстве известных баз данных тестовых примеров Juliet Национального
института стандартизации и технологий США. % link
% Add picture with Shebang and encoding.

База данных тестовых примеров должна состоять из двух частей. Первая часть -- тестовые примеры. Вторая часть -- описание тестовых примеров. 
Элемент второй части должен предоставлять всю необходимую информацию по тестовому примеру в виде файла или группы файлов, организованных в структуру каталогов. Файлы описатели, они же файлы аннотации, не должны зависеть от языка программирования исходного кода инструмента статического анализа и языка программирования анализируемых программ. 
Файл аннотация -- это файл в JSON формате, который описывает тестовый пример как для содержащего, так и для не содержащего ошибку.

Фреймворк должен сравнивать статические анализаторы по величине полноты проведенного на тестовом наборе анализа.
Для этого фреймворк должен поддерживать запуск нескольких статических анализаторов в одной связке. 

Фреймворк не должен зависеть от операционной системы. 

Существуют инструменты статического анализа программ для множества
разных языков, иногда сразу нескольких. Acceptance Testing Framework не должен
зависеть от целевого языка анализируемых программ. Он должен подходить для
тестирования анализаторов c такими целевыми языками программирования как C,
C++, Java, C\#, Python и других языков.

На вход фреймворк принимает базу данных с тестовыми примерами и файлами аннотациями -- тестовый набор. Фреймворк не должен менять структуру или отдельные файлы тестового набора.
Совместимость достигается за счет соблюдения в тестовом наборе правил, которые
накладывает фреймворк.

Тестовый набор может содержать тестовые примеры трех видов. Первые –
тестовые примеры с ошибкой, дефектом, о котором должен предупредить инструмент статического анализа. 
Вторые – не содержащие ошибку тестовые примеры. 
Третьи – тестовые примеры, проверяющие поддержку специфических конструкций языка программирования. 
Первая и вторая группы это соответственно истинно положительные и ложноположительные предупреждения
(третья группа также ложноположительные предупреждения). Для определения
достоверного качества инструмента статического анализа необходима возможность проверки всех групп тестовых примеров, выделение отдельной статистики
для каждой группы.

Средствами статического анализа возможно проверить исходную программу на наличие в ней уязвимостей безопасности, ошибок времени выполнения,
несоответствия стилю форматирования и комментариев. Фреймворк должен поддерживать неограниченное количество средств проверки правил кодирования,
включая, помимо прочего, стили форматирования и комментариев.

Результат запуска инструмента статического анализа на исходном коде возможен в двух вариантах. Первый вариант это отсутствие предупреждения, выдаваемого статическим анализатором. Второй вариант, статический анализатор, просканировав исходный код, выдает предупреждение. Вариант, в котором выдается
предупреждение, может быть представлен текстом, напечатанным в стандартном потоке вывода или ошибки, или файлом определенного формата. Если инструмент статического анализа встроен в среду разработки, то результат его работы есть подсветка кода
в местах, соответствующих выданным предупреждениям. Необходимо, чтобы
независимо от выдаваемого формата фреймворк поддерживал возможность запуска и оценки результата работы инструмента статического анализа.

Результат запуска инструмента статического анализа на тестовом наборе фреймворк должен
агрегировать во внутреннее представление. На основе внутреннего представления фреймворк должен формировать отчет для вывода результата и представления данного результата пользователю в различных форматах: машиночитаемый
(JSON, XML и другие), вывод, отформатированный для отображения результата на экране, формат HTML. А также возможность расширения списка форматов
отчета по запросу.

\section{Требования к тестовому примеру}
Тестовый пример должен содержать только одну ошибку, о которой следует выдать предупреждение статическому анализатору. Или не содержать ошибок вообще.

Чтобы определить, является ли ошибка, сообщаемая статическим анализаторов, верной, необходимо знать, где именно в исходном коде тестового примера находится ошибка. Требуется точное описание места ошибки в тестовом примере. 




\FloatBarrier
           % Глава 2
\chapter{}\label{ch:ch3}

\section{}\label{sec:ch3/sect1}

\clearpage
           % Глава 3
\chapter*{Заключение}                       % Заголовок
\addcontentsline{toc}{chapter}{Заключение}  % Добавляем его в оглавление

%% Согласно ГОСТ Р 7.0.11-2011:
%% 5.3.3 В заключении диссертации излагают итоги выполненного исследования, рекомендации, перспективы дальнейшей разработки темы.
%% 9.2.3 В заключении автореферата диссертации излагают итоги данного исследования, рекомендации и перспективы дальнейшей разработки темы.
%% Поэтому имеет смысл сделать эту часть общей и загрузить из одного файла в автореферат и в диссертацию:
      % Заключение
% \chapter*{Список сокращений и условных обозначений} % Заголовок
\addcontentsline{toc}{chapter}{Список сокращений и условных обозначений}  % Добавляем его в оглавление
% при наличии уравнений в левой колонке значение параметра leftmargin приходится подбирать вручную

        % Список сокращений и условных обозначений
% \chapter*{Словарь терминов}             % Заголовок
\addcontentsline{toc}{chapter}{Словарь терминов}  % Добавляем его в оглавление

\textbf{TeX} : Cистема компьютерной вёрстки, разработанная американским профессором информатики Дональдом Кнутом

      % Словарь терминов
\include{Dissertation/references}      % Список литературы
% \include{Dissertation/lists}           % Списки таблиц и изображений (иллюстративный материал)

\setcounter{totalchapter}{\value{chapter}} % Подсчёт количества глав

%%% Настройки для приложений
\appendix
% Оформление заголовков приложений ближе к ГОСТ:
\setlength{\midchapskip}{20pt}
\renewcommand*{\afterchapternum}{\par\nobreak\vskip \midchapskip}
\renewcommand\thechapter{\Asbuk{chapter}} % Чтобы приложения русскими буквами нумеровались

% \chapter{Примеры вставки листингов программного кода}\label{app:A}
        % Приложения

\setcounter{totalappendix}{\value{chapter}} % Подсчёт количества приложений

\end{document}
