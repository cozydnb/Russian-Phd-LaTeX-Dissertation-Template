\chapter{Требования}\label{ch:ch2}

\section{Выбор характеристик для оценки качества инструментов статического анализа}\label{sec:ch2/sec1}
Результат запуска статического анализатора на тестовом наборе или проекте это множество верных и ложных предупреждений об ошибках (соответственно
true positives -- TP, false positives -- FP). Ложное предупреждение об ошибке свидетельствует о наличие неточности в алгоритме статического анализатора. Большое
количество ложных предупреждений -- это шум в результатах работы статического анализатора и признак низкого качества проводимого анализа. Большое
количество верных предупреждений при малом количестве ложных предупреждений свидетельствует о высоком качестве проводимого анализа и о его полноте.
Под полнотой понимается способность статического анализатора обнаруживать
как можно больше дефектов, являющихся истинными ошибками в программе. Отношение верных предупреждений об ошибках
к общему количеству предупреждений это метрика -- точность (precision) \eqref{eq:precision}. 
\begin{equation}\label{eq:precision}
precision = TP/(TP + FP)
\end{equation}
Данная метрика используется в статистике для оценки результатов исследования на основе полученных true positives и false positives. Данная метрика широко распространена в
оценке качества статических анализаторов. 
Общее количество предупреждений
формируется из суммы верных и ложных предупреждений. Ситуации, в которых инструмент не выдал предупреждения, однако ошибка имеет место быть,
являются false negatives. Сумма false negatives и true positives это общее количество ошибок, которое содержится в тестовом наборе или коде проекта. Чаще
эта величина вычисляется для статических анализаторов, сканирующих тестовые наборы, так как определение точного количества ошибок в программном
проекте –- сложная задача. Отношение количества верных предупреждений, true positives, к
общему количеству ошибок в тестовом наборе или проекте это мера оценки качества результата инструмента статического анализа, называемая отклик (recall) \eqref{eq:recall}. 
\begin{equation}\label{eq:recall}
recall = TP/(TP+FN)
\end{equation}
Данная мера так же как и точность используется в статистике для оценки качества проводимого
исследования. Идеальный инструмент статического анализа не выдает ложных
предупреждений и сообщает о всех существующих ошибках. Для идеального инструмента значения точности и отклика равны единице. Теоретически полностью
противоположный идеальному инструменту инструмент выдает только ложные
предупреждения, шум. Для такого инструмента значения точности и отклика равны
нолю. Таким образом значения точности и отклика находятся в диапазоне от ноля до
единицы. На основе значений точности и отклика вычисляется величина F-measure.
F-measure характеризует точность проводимого исследования. 
F-measure вычисляется, как гармоническое среднее точности и отклика \eqref{eq:fmeasure}. 
\begin{equation}\label{eq:fmeasure}
F-measure = 2 * precision * recall / (precision + recall)
\end{equation}
Наибольшее возможное значение для F-measure равно одному, соответсвующее идеальным значениям точности и отклика.
Наименьшее возможное значение для F-measure равно нолю, в данном случае одно из величин, точность или отклик равно нолю.

Чтобы измерить величины точность, отклик, F-measure для участвующего в исследовании статического анализатора
предлагается следующий порядок действий. Во-первых, сформировать тестовый набор. Данный тестовый набор должен соответствовать инспекциям (чекерам) статического
анализатора, ошибкам которые он обнаруживает. Во-вторых, запустить статический анализатор на данном тестовом наборе. Данный шаг осуществляется
средствами Acceptance Testing Framework, или сокращенно ATF. В-третьих, собрать статистику работы статического анализатора
на тестовом наборе. Данный шаг осуществляется средствами ATF. В-четвертых,
на основе полученной статистики, которая есть количество верных и ложных
предупреждений, вычислить значения точности и отклика. 

Результаты измерений будут
представлены в двух интерпретациях. Первая интерпретация -- значения величин
точность и отклик для тестового набора, на котором проводился анализ для данного статического анализатора, то есть проверяется качество результата инструмента в рамках заявленных возможностей обнаружения определённого типа ошибок. Вторая интерпретация –- значения величин 
точность и отклик на тестовых примерах, представляющих полное количество возможных обнаруживаемых дефектов в рамках тестового набора. Вторая интерпретация необходима для объективного сравнения качества
анализа, проводимого статическими анализаторами.

\section{Требования к Acceptance Testing Framework}\label{sec:ch2/sect2}
Инструменты статического анализа исходного кода должны проверять состояние исходного кода программ с точки зрения очень разных правил, которые
могут применяться в качестве промышленного или общекорпоративного стандарта кодирования. Несмотря на то, что современные статические анализаторы
исходного кода уделяют особое внимание безопасности кода, отсутствию логических ошибок и производительности, некоторые правила кодирования, применяемые в компаниях или отрасли, могут содержать такие требования к коду, как
стиль отступов, соглашения об именах и т. д. Например, если рассмотреть
программы, написанные на языке программирования Python, то исходный код может содержать комментарии определенного вида, такие как Shebang, кодировка файла, информация о версии или лицензии. 
Вот почему, пытаясь удовлетворить потребности тестирования промышленных статических анализаторов исходного кода, такой фреймворк не может полагаться на специальные комментарии и форматирование кода, как, например, используемые в большинстве известных баз данных тестовых примеров Juliet Национального
института стандартизации и технологий США. % link
% Add picture with Shebang and encoding.

База данных тестовых примеров должна состоять из двух частей. Первая часть -- тестовые примеры. Вторая часть -- описание тестовых примеров. 
Элемент второй части должен предоставлять всю необходимую информацию по тестовому примеру в виде файла или группы файлов, организованных в структуру каталогов. Файлы описатели, они же файлы аннотации, не должны зависеть от языка программирования исходного кода инструмента статического анализа и языка программирования анализируемых программ. 
Файл аннотация -- это файл в JSON формате, который описывает тестовый пример как для содержащего, так и для не содержащего ошибку.

Фреймворк должен сравнивать статические анализаторы по величине полноты проведенного на тестовом наборе анализа.
Для этого фреймворк должен поддерживать запуск нескольких статических анализаторов в одной связке. 

Фреймворк не должен зависеть от операционной системы. 

Существуют инструменты статического анализа программ для множества
разных языков, иногда сразу нескольких. Acceptance Testing Framework не должен
зависеть от целевого языка анализируемых программ. Он должен подходить для
тестирования анализаторов c такими целевыми языками программирования как C,
C++, Java, C\#, Python и других языков.

На вход фреймворк принимает базу данных с тестовыми примерами и файлами аннотациями -- тестовый набор. Фреймворк не должен менять структуру или отдельные файлы тестового набора.
Совместимость достигается за счет соблюдения в тестовом наборе правил, которые
накладывает фреймворк.

Тестовый набор может содержать тестовые примеры трех видов. Первые –
тестовые примеры с ошибкой, дефектом, о котором должен предупредить инструмент статического анализа. 
Вторые – не содержащие ошибку тестовые примеры. 
Третьи – тестовые примеры, проверяющие поддержку специфических конструкций языка программирования. 
Первая и вторая группы это соответственно истинно положительные и ложноположительные предупреждения
(третья группа также ложноположительные предупреждения). Для определения
достоверного качества инструмента статического анализа необходима возможность проверки всех групп тестовых примеров, выделение отдельной статистики
для каждой группы.

Средствами статического анализа возможно проверить исходную программу на наличие в ней уязвимостей безопасности, ошибок времени выполнения,
несоответствия стилю форматирования и комментариев. Фреймворк должен поддерживать неограниченное количество средств проверки правил кодирования,
включая, помимо прочего, стили форматирования и комментариев.

Результат запуска инструмента статического анализа на исходном коде возможен в двух вариантах. Первый вариант это отсутствие предупреждения, выдаваемого статическим анализатором. Второй вариант, статический анализатор, просканировав исходный код, выдает предупреждение. Вариант, в котором выдается
предупреждение, может быть представлен текстом, напечатанным в стандартном потоке вывода или ошибки, или файлом определенного формата. Если инструмент статического анализа встроен в среду разработки, то результат его работы есть подсветка кода
в местах, соответствующих выданным предупреждениям. Необходимо, чтобы
независимо от выдаваемого формата фреймворк поддерживал возможность запуска и оценки результата работы инструмента статического анализа.

Результат запуска инструмента статического анализа на тестовом наборе фреймворк должен
агрегировать во внутреннее представление. На основе внутреннего представления фреймворк должен формировать отчет для вывода результата и представления данного результата пользователю в различных форматах: машиночитаемый
(JSON, XML и другие), вывод, отформатированный для отображения результата на экране, формат HTML. А также возможность расширения списка форматов
отчета по запросу.

\section{Требования к тестовому примеру}
Тестовый пример должен содержать только одну ошибку, о которой следует выдать предупреждение статическому анализатору. Или не содержать ошибок вообще.

Чтобы определить, является ли ошибка, сообщаемая статическим анализаторов, верной, необходимо знать, где именно в исходном коде тестового примера находится ошибка. Требуется точное описание места ошибки в тестовом примере. 




\FloatBarrier
